\documentclass[12pt]{article}

\usepackage[a4paper, total={6in, 8in}]{geometry}

\usepackage[utf8]{inputenc}
\usepackage[T1]{fontenc}
% \usepackage[serbian]{babel}

\usepackage{graphicx, float}
\graphicspath{{images/}}

\renewcommand{\baselinestretch}{1.2} % Line spacing

% Parskip and parindent
\setlength{\parindent}{0pt} % Begin of paragraph indentation
\setlength{\parskip}{1em} % Paragraph spacing

%--- for references ---------
\usepackage[
   backend=biber,
   style=numeric
   ]{biblatex}

\addbibresource{bibliography.bib}
%----------------------------

\begin{document}

   % Title Page
   \newgeometry{top=1in, bottom=1in, left=1in, right=1in} % New margins for title page
   \begin{titlepage}
      \begin{center}
         
         % add your university logo here
         % negative value moves the logo up
         \vspace*{-1in}
         \includegraphics[width=0.4\textwidth]{raf_logo.png}

         % set font size to 14pt
         \vspace{1in}
         \Large
         \textbf{DIPLOMSKI RAD}
         
         % set horizontal margin for the title to 1.5in and center it
         \vspace{1in}
         \Huge
         \textbf{Slika je vredna 16x16 reči: \\ Vision Transformeri}
         
         \vspace{1in}


         \fontsize{14pt}{18pt}\selectfont
         \textbf{Vanja Kovinić} \\
         \textbf{RN 42/2020}
         \vspace*{1.5in}
         
         \begin{center}
            \normalsize
            \begin{tabular}{p{0.7\textwidth} p{0.5\textwidth}}
               \fontsize{14pt}{18pt}\selectfont   
               \textbf{Mentor:} & 
            
               \fontsize{14pt}{18pt}\selectfont
               \textbf{Komisija:} \\
               dr Nemanja Ilić & dr Nemanja Ilić \\
                                 & dr Nevena Marić \\
            \end{tabular}
         \end{center}

         \vspace*{\fill}

         \normalsize
         Beograd, septembar 2024.


         
      \end{center}
   \end{titlepage}
   \restoregeometry % Restore original margins

   \newpage


   % Table of Contents
   \renewcommand{\contentsname}{Sadržaj}
   \tableofcontents
   
   \newpage
   
   \thispagestyle{empty} % Remove page number from Abstract page

   % Define a command to format a specific section title
   \newcommand{\specialsection}[1]{
      \section*{\centering{#1}} % Center and italicize the section title
   }

   \vspace*{0.5in}
   \specialsection{Apstrakt}
   

   \vspace*{0.5in}

   Ovaj diplomski rad istražuje \textbf{Vision Transformere} (\textbf{ViT}),
   nov pristup u oblasti računarskog vida koji koristi arhitekturu
   transformera prvobitno razvijenu za obradu prirodnog jezika. Prvi deo rada pruža detaljan pregled arhitekture transformera, 
   uključujući ključne komponente kao što su \textbf{\textit{self-attention} mehanizam} i \textbf{poziciono enkodovanje},
   i diskutuje njihove svrhe i funkcionalnosti. Nakon toga, fokus se prebacuje
   na Vision Transformere, objašnjavajući kako se slike transformišu
   u \textbf{tokene} i obrađuju kroz \textbf{enkoder transformera} kako bi se primenili na rešavanje vizuelnih zadataka.

   Rad zatim ulazi u praktične aspekte implementacije Vision Transformera,
   uključujući izbor i podešavanje \textbf{hiperparametara} za poboljšanje performansi.
   Izvršeno je i poređenje sa referentnim implementacijama, i predložen pristup za 
   poboljšanje performansi. Prikazani su različiti
   eksperimenti, zajedno sa diskusijom njihovih rezultata, pružajući uvid
   u efikasnost i izazove povezane sa Vision Transformerima.

   Na kraju, rad naglašava značaj Vision Transformera u oblasti računarskog vida, 
   prikazujući njihov potencijal i ograničenja, kao i njihove praktične primene.

   \newpage
   \pagenumbering{arabic}
   \setcounter{page}{1}

   \section{Uvod}
   
   \textit{"Pre otprilike 540 miliona godina, 
   Zemlja je bila obavijena tamom.
   Ovo nije bilo zbog nedostatka svetlosti,
   već zato što organizmi još uvek nisu razvili sposobnost da vide.
   Iako je sunčeva svetlost mogla da prodre u okeane do dubine
   od 1.000 metara i hidrotermalni izvori na dnu mora isijavali 
   svetlost u kojoj je život cvetao, nijedno oko nije se moglo naći 
   u tim drevnim okeanima, nijedna retina, rožnjača ili sočivo. 
   Sva svetlost i život nikada nisu viđeni. Koncept gledanja nije ni postojao tada 
   i ova sposobnost nije ostvarena sve dok nije stvorena.
   }
   
   \textit{Iz nama nepoznatih razloga, trilobiti su se pojavili kao prva bića sposobna
   da spoznaju svetlost. Oni su prvi prepoznali da postoji nešto izvan
   njih samih, svet okružen višestrukim jedinkama. Rađenje vida se smatra da je pokrenulo
   kambrijsku eksploziju, period u kojem se veliki broj vrsta životinja pojavljuje u 
   fosilnom zapisu. Vid je započeo kao pasivno iskustvo, jednostavno propuštanje svetlosti, 
   ali je ubrzo postao aktivniji. Nervni sistem je počeo da evoluira, vid je prešao u uvid, 
   gledanje je postalo razumevanje, a razumevanje je dovelo do akcije, a sve to je dovelo do 
   nastanka inteligencije.}
   
   \textit{Danas nismo više zadovoljni vizuelnom spoznajom koju nam je priroda dala. 
   Radoznalost nas je navela da stvorimo mašine koje mogu da "vide" kao mi, pa čak i inteligentnije."} - Li Fei-Fei \cite{li_fei_fei}
   

   \subsection{Istorija i motivacija}
   \subsubsection{Rani Razvoj u Računarskom Vidu}

   Koreni računarskog vida potiču iz ranih pokušaja da se razume 
   i interpretira vizuelni podatak korišćenjem matematičkih modela i računara. 
   U početku, istraživanja u oblasti računarskog vida fokusirala su se na 
   jednostavne zadatke kao što su detekcija ivica, prepoznavanje objekata i 
   osnovna obrada slika. Rane metode su se u velikoj meri oslanjale na ručnu 
   izradu karakteristika (engl. \textbf{\textit{features}}) slike i algoritme dizajnirane da imitiraju osnovne aspekte 
   ljudskog vida.

   \subsubsection{Uspon Dubokog Učenja}

   Značajan preokret u računarskom vidu dogodio se sa pojavom \textbf{dubokog učenja}. 
   \textbf{Konvolucione Neuronske Mreže} (\textbf{CNNs}), koje su predstavili Yann LeCun i drugi \cite{lecun_cnn} 
   krajem 1980-ih i početkom 1990-ih, revolucionisale su ovu oblast uvođenjem automatskog 
   ekstraktovanja karakteristika kroz slojeve koji se uče (engl. \textbf{\textit{learnable features}}). \textbf{CNN}-ovi su pokazali izuzetne 
   performanse u različitim zadacima klasifikacije slika, omogućavajući računarima da 
   nauče složene reprezentacije vizuelnih podataka. Ovo otkriće je kasnije propaćeno uspehom modela 
   kao što su \textbf{AlexNet} \cite{alexnet}, \textbf{VGGNet} \cite{vgg} i \textbf{ResNet} \cite{resnet}, koji su postavili nove standarde u izazovima 
   prepoznavanja slika.

   \subsubsection{Ograničenja CNN-ova}

   Uprkos svom uspehu, \textbf{CNN}-ovi imaju inherentna ograničenja 
   koja su motivisala potragu za novim pristupima. Jedan od 
   značajnih nedostataka je njihova poteškoća u povezivanju udaljenih zavisnosti
   i globalnog konteksta unutar slike. \textbf{CNN}-ovi obično obrađuju slike kroz seriju 
   lokalizovanih konvolucionih operacija, što može ograničiti njihovu sposobnost da 
   razumeju odnose između udaljenih elemenata na slici.

   \subsubsection{Motivacija za uvođenje Vision Transformera}

   Pojava \textbf{Vision Transformera} (\textbf{ViT}) predstavlja odgovor na ova ograničenja. 
   Inspirisani uspehom modela transformera u obradi prirodnog jezika (\textbf{NLP}), 
   istraživači su pokušali da primene iste principe u računarskom vidu. 
   Transformeri koriste \textbf{\textit{self-attention}} mehanizam za povezivanje globalnih zavisnosti, 
   što ih čini pogodnim za zadatke koji zahtevaju razumevanje 
   složenih odnosa unutar vizuelnih podataka.
   
   \newpage

   \textbf{Vision Transformeri} rešavaju nekoliko izazova sa kojima se suočavaju \textbf{CNN}-ovi. 
   Pretvaranjem slika u sekvence parčića (engl. \textbf{\textit{image patches}}) i primenom \textbf{\textit{self-attention}} mehanizma preuzetim iz \textbf{transformera}, 
   ViTs mogu efikasnije modelovati globalni kontekst. 
   Ovaj pristup omogućava \textbf{ViT}-ovima da postignu vrhunske performanse na različitim  
   testovima klasifikacije slika i pokazuje njihov potencijal da unaprede oblast računarskog vida.

   \newpage
   \printbibliography


\end{document}