\documentclass[12pt]{article}

\usepackage[a4paper, total={6in, 8in}]{geometry}

\usepackage[utf8]{inputenc}
\usepackage[T1]{fontenc}
% \usepackage[serbian]{babel}

\usepackage{graphicx, float}
\graphicspath{{images/}}

\renewcommand{\baselinestretch}{1.15} % Line spacing

% Parskip and parindent
\setlength{\parindent}{0pt} % Begin of paragraph indentation
\setlength{\parskip}{1em} % Paragraph spacing

\begin{document}

   % Title Page
   \newgeometry{top=1in, bottom=1in, left=1in, right=1in} % New margins for title page
   \begin{titlepage}
      \begin{center}
         
         % add your university logo here
         % negative value moves the logo up
         \vspace*{-1in}
         \includegraphics[width=0.4\textwidth]{raf_logo.png}

         % set font size to 14pt
         \vspace{1in}
         \Large
         \textbf{DIPLOMSKI RAD}
         
         % set horizontal margin for the title to 1.5in and center it
         \vspace{1in}
         \Huge
         \textbf{Slika je vredna 16x16 reči: \\ Vision Transformeri}
         
         \vspace{1in}


         \fontsize{14pt}{18pt}\selectfont
         \textbf{Vanja Kovinić} \\
         \textbf{RN 42/2020}
         \vspace*{1.5in}
         
         \begin{center}
            \normalsize
            \begin{tabular}{p{0.7\textwidth} p{0.5\textwidth}}
               \fontsize{14pt}{18pt}\selectfont   
               \textbf{Mentor:} & 
            
               \fontsize{14pt}{18pt}\selectfont
               \textbf{Komisija:} \\
               dr Nemanja Ilić & dr Nemanja Ilić \\
                                 & dr Nevena Marić \\
            \end{tabular}
         \end{center}

         \vspace*{\fill}

         \normalsize
         Beograd, septembar 2024.


         
      \end{center}
   \end{titlepage}
   \restoregeometry % Restore original margins

   \newpage
   
   \thispagestyle{empty} % Remove page number from Abstract page

   % Define a command to format a specific section title
   \newcommand{\specialsection}[1]{
      \section*{\centering{#1}} % Center and italicize the section title
      \addcontentsline{toc}{section}{#1} % Optional: Add the section to the table of contents
   }

   \vspace*{0.5in}
   \specialsection{Apstrakt}
   

   \vspace*{0.5in}

   Ovaj diplomski rad istražuje \textbf{Vision Transformere} (\textbf{ViT}),
   nov pristup u oblasti računarskog vida koji koristi arhitekturu
   transformera prvobitno razvijenu za obradu prirodnog jezika. Prvi deo rada pruža detaljan pregled arhitekture transformera, 
   uključujući ključne komponente kao što su \textbf{\textit{self-attention} mehanizam} i \textbf{poziciono enkodovanje},
   i diskutuje njihove svrhe i funkcionalnosti. Nakon toga, fokus se prebacuje
   na Vision Transformere, objašnjavajući kako se slike transformišu
   u \textbf{tokene} i obrađuju kroz \textbf{enkoder transformera} kako bi se primenili na rešavanje vizuelnih zadataka.

   Rad zatim ulazi u praktične aspekte implementacije Vision Transformera,
   uključujući izbor i podešavanje \textbf{hiperparametara} za poboljšanje performansi.
   Izvršeno je i poređenje sa referentnim implementacijama, i predložen pristup za 
   poboljšanje performansi. Prikazani su različiti
   eksperimenti, zajedno sa diskusijom njihovih rezultata, pružajući uvid
   u efikasnost i izazove povezane sa Vision Transformerima.

   Na kraju, rad naglašava značaj Vision Transformera u oblasti računarskog vida, 
   prikazujući njihov potencijal i ograničenja, kao i njihove praktične primene.


   
\end{document}